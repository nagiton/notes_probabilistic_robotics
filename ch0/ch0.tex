\documentclass{jarticle}

\evensidemargin 0.0in
\oddsidemargin 0.0in
\topmargin 0.0in
\textwidth 15.9cm
\textheight 8.0in
\headheight 0.0in
\headsep 0.0in
\topskip 0.0in
\textheight 9.0in
\renewcommand{\baselinestretch}{1.2}

\def\vec#1{\mbox{\boldmath $#1$}}

\begin{document} 
\setcounter{section}{-1}
\section{定義、公式、約束事}
\subsection{アインシュタインの縮約}
本ノートではアインシュタインの縮約記法を用いる。

すなわち、同じ添字がついた文字の積は、その添字について和をとっているものとみなす。

例を挙げると、$\vec{x}=(x_1, x_2,x_3)$というベクトルと$\vec{y}=(y_1,y_2,y_3)$というベクトルの内積は以下のようになる。

\begin{eqnarray}
\vec{x} \cdot \vec{y} = x_i y_i = \sum _{i=1} ^3 x_i y_i
\end{eqnarray}

記法になれるまではスムーズに読めないかもしれないが、この記法を使わないと行列やベクトルの積を成分表示したときに和の記号がたくさん出てきてしまい、かえって混乱を招くように思う。

また、この記法を使うと、あえて和をとらない$ x_i y_i $がほしいときどうするのか疑問に思うかもしれない。その場合はクロネッカーのデルタを用いて$  \delta _{ij} x_j y_j$などとすればよい。

\end{document}