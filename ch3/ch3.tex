\documentclass{jarticle}
\usepackage{amsmath,amssymb}


\evensidemargin 0.0in
\oddsidemargin 0.0in
\topmargin 0.0in
\textwidth 15.9cm
\textheight 8.0in
\headheight 0.0in
\headsep 0.0in
\topskip 0.0in
\textheight 9.0in
\renewcommand{\baselinestretch}{1.2}

\def\vec#1{\mbox{\boldmath $#1$}}

\begin{document} 
\setcounter{section}{3}
\setcounter{subsection}{2}
\setcounter{subsubsection}{3}
\subsubsection{カルマンフィルタ(KF)の数学的導出}
まずは予測ステップの導出から始める。1ステップ前の信念$bel(x_{t-1})$に、制御$u_t$が加わったあとの信念$\overline{bel}(x_t) $が計算できる。
\begin{eqnarray}
\overline{bel}(x_t) &=& \int p(x_t | x_{t-1}, u_t) bel(x_{t-1}) dx_{t-1} \\
&=&\int \mathcal{N}(x_t; A_t x_{t-1} + B_t u_t, R_t) \mathcal{N}(x_{t-1}; \mu _{t-1}, \Sigma _{t-1}) dx_{t-1}
\end{eqnarray}

ただし、$ \mathcal{N}(x; \mu, \Sigma )$は平均$\mu$、共分散行列$\Sigma$のガウス分布の$x$における確率密度を表す。具体的にガウス分布で表示すると

\begin{eqnarray}
\overline{bel}(x_t) &=&\int \mathcal{N}(x_t; A_t x_{t-1} + B_t u_t, R_t) \mathcal{N}(x_{t-1}; \mu _{t-1}, \Sigma _{t-1}) dx_{t-1}\\
&=& \eta \int \exp[-\frac{1}{2}(x_t - (A_t x_{t-1} + B_t u_t))^T R_t ^{-1}(x_t - (A_t x_{t-1} + B_t u_t)) ] \nonumber \\
&& \hspace{3cm} \exp[-\frac{1}{2} (x_{t-1}-\mu _{t-1})^T  \Sigma ^{-1}  (x_{t-1}-\mu _{t-1})]dx_{t-1}
\end{eqnarray}

指数関数の積は肩の和を取ればよい。なので以下のように関数$L_t$を使って書き換えられる。

\begin{eqnarray}
\overline{bel}(x_t) &=& \eta \int \exp[-L_t] dx_{t-1}\\
 L_t &=&\frac{1}{2}(x_t - (A_t x_{t-1} + B_t u_t))^T R_t ^{-1}(x_t - (A_t x_{t-1} + B_t u_t)) \nonumber  \\
 && \hspace{3cm}+\frac{1}{2} (x_{t-1}-\mu _{t-1})^T  \Sigma ^{-1}  (x_{t-1}-\mu _{t-1})
\end{eqnarray}

$x_{t-1}$に関する積分を実行するために、$L_t$を$x_{t-1}$によらない項のみを含む部分$L_t (x_t)$と、$x_{t-1}$による項も含む部分$L_t (x_{t-1},x_t)$に分解する($L_t (x_{t-1},x_t)$は$x_{t-1}$によらない項を含んでもよい、なので一意な分解ではない)。こうすると、この積分は

\begin{eqnarray}
\overline{bel}(x_t) &=& \eta L_t (x_t)  \int \exp L_t (x_{t-1},x_t)dx_{t-1}\\
\end{eqnarray}

とかける。この積分だが、$L_t$が$x_{t-1}$に関して2次式だったことを思い出すと、$x_{t-1}$に関して「平方完成」すればある$x_{t-1}$によらないベクトル$\xi$と行列$\Psi$を用いて、$(x_{t-1}-\xi)^T \Psi^{-1} (x_{t-1}-\xi)$の形に変形できそうである。この形に変形できれば、$\exp L_t (x_{t-1},x_t)$の積分は多変量正規分布$\mathcal{N}(x_{t-1};\xi,\Psi )$の規格化定数を求める積分と全く同じになり、特に$x_t$によらない定数になる。

この変形を行うため、あえて$L_t (x_{t-1},x_t)$は$x_{t-1}$によらない項を含めても良いことにしている。

$\exp L_t (x_{t-1},x_t)$の積分を多変量正規分布$\mathcal{N}(x_{t-1};\xi,\Psi )$の積分の形にもっていくために、この分布$\exp L_t (x_{t-1},x_t)$の頂点の座標$\xi$を求める。頂点$\xi$では、$L_t (x_{t-1},x_t)$はベクトル$x_{t-1}$のあらゆる方向に対して微分係数が0になっている。なので、$L_t (x_{t-1},x_t)$を$x_{t-1}$の各成分に関して微分したもの(この操作を「$L_t$を$x_{t-1}$で微分」という)がすべて0となっている。

$x_{t-1}$の第$i$成分を$x_{i,t-1}$とし、実際にこの計算を実行すると

\begin{eqnarray}
\frac{\partial  L_t}{\partial x_{i,t-1}}  &=&\frac{\partial }{\partial x_{i,t-1}}   \frac{1}{2}(x_t - (A_t x_{t-1} + B_t u_t))^T R_t ^{-1}(x_t - (A_t x_{t-1} + B_t u_t)) \nonumber  \\
 && \hspace{3cm}+\frac{\partial }{\partial x_{i,t-1}}  \frac{1}{2} (x_{t-1}-\mu _{t-1})^T  \Sigma ^{-1}  (x_{t-1}-\mu _{t-1}) \\
 &=&
\end{eqnarray}




\end{document}