\documentclass{jarticle}
\usepackage{amsmath,amssymb}


\evensidemargin 0.0in
\oddsidemargin 0.0in
\topmargin 0.0in
\textwidth 15.9cm
\textheight 8.0in
\headheight 0.0in
\headsep 0.0in
\topskip 0.0in
\textheight 9.0in
\renewcommand{\baselinestretch}{1.2}

\def\vec#1{\mbox{\boldmath $#1$}}

\begin{document} 
\setcounter{section}{3}
\setcounter{subsection}{2}
\setcounter{subsubsection}{3}
\subsubsection{カルマンフィルタ(KF)の数学的導出}
まずは予測ステップの導出から始める。1ステップ前の信念$bel(x_{t-1})$に、制御$u_t$が加わったあとの信念$\overline{bel}(x_t) $が計算できる。
\begin{eqnarray}
\overline{bel}(x_t) &=& \int p(x_t | x_{t-1}, u_t) bel(x_{t-1}) dx_{t-1} \\
&=&\int \mathcal{N}(x_t; A_t x_{t-1} + B_t u_t, R_t) \mathcal{N}(x_{t-1}; \mu _{t-1}, \Sigma _{t-1}) dx_{t-1}
\end{eqnarray}

ただし、$ \mathcal{N}(x; \mu, \Sigma )$は平均$\mu$、共分散行列$\Sigma$のガウス分布の$x$における確率密度を表す。具体的にガウス分布で表示すると

\begin{eqnarray}
\overline{bel}(x_t) &=&\int \mathcal{N}(x_t; A_t x_{t-1} + B_t u_t, R_t) \mathcal{N}(x_{t-1}; \mu _{t-1}, \Sigma _{t-1}) dx_{t-1}\\
&=& \eta \int \exp[-\frac{1}{2}(x_t - (A_t x_{t-1} + B_t u_t))^T R_t ^{-1}(x_t - (A_t x_{t-1} + B_t u_t)) ] \nonumber \\
&& \hspace{3cm} \exp[-\frac{1}{2} (x_{t-1}-\mu _{t-1})^T  \Sigma ^{-1}  (x_{t-1}-\mu _{t-1})]dx_{t-1}
\end{eqnarray}

指数関数の積は肩の和を取ればよい。なので以下のように関数$L_t$を使って書き換えられる。

\begin{eqnarray}
\overline{bel}(x_t) &=& \eta \int \exp[-L_t] dx_{t-1}\\
 L_t &=&\frac{1}{2}(x_t - (A_t x_{t-1} + B_t u_t))^T R_t ^{-1}(x_t - (A_t x_{t-1} + B_t u_t)) \nonumber  \\
 && \hspace{3cm}+\frac{1}{2} (x_{t-1}-\mu _{t-1})^T  \Sigma ^{-1}  (x_{t-1}-\mu _{t-1})
\end{eqnarray}

$x_{t-1}$に関する積分を実行するために、$L_t$を$x_{t-1}$によらない項のみを含む部分$L_t (x_t)$と、$x_{t-1}$による項も含む部分$L_t (x_{t-1},x_t)$に分解する($L_t (x_{t-1},x_t)$は$x_{t-1}$によらない項を含んでもよい、なので一意な分解ではない)。こうすると、この積分は

\begin{eqnarray}
\overline{bel}(x_t) &=& \eta \exp[-L_t (x_t)]  \int \exp [-L_t (x_{t-1},x_t)]dx_{t-1}\\
\end{eqnarray}

とかける。この積分だが、$L_t$が$x_{t-1}$に関して2次式だったことを思い出すと、$x_{t-1}$に関して「平方完成」すればある$x_{t-1}$によらないベクトル$\xi$と行列$\Psi$を用いて、$\frac{1}{2}(x_{t-1}-\xi)^T \Psi^{-1} (x_{t-1}-\xi)$の形に変形できそうである。この形に変形できれば、$\exp L_t (x_{t-1},x_t)$の積分は多変量正規分布$\mathcal{N}(x_{t-1};\xi,\Psi )$の規格化定数を求める積分と全く同じになり、特に$x_t$によらない定数になる。

この変形を行うため、あえて$L_t (x_{t-1},x_t)$は$x_{t-1}$によらない項を含めても良いことにしている。

$\exp L_t (x_{t-1},x_t)$の積分を多変量正規分布$\mathcal{N}(x_{t-1};\xi,\Psi )$の積分の形にもっていくために、この分布$\exp L_t (x_{t-1},x_t)$の頂点の座標$\xi$を求める。頂点$\xi$では、$L_t (x_{t-1},x_t)$はベクトル$x_{t-1}$のあらゆる方向に対して微分係数が0になっている。なので、$L_t (x_{t-1},x_t)$を$x_{t-1}$の各成分に関して微分したもの(この操作を「$L_t$を$x_{t-1}$で微分」という)がすべて0となっている。

$x_{t-1}$の第$i$成分を$x_{i,t-1}$とし、実際にこの計算を実行すると

\begin{eqnarray}
\frac{\partial  L_t}{\partial x_{i,t-1}}  &=&\frac{\partial }{\partial x_{i,t-1}}   \frac{1}{2}(x_t - (A_t x_{t-1} + B_t u_t))^T R_t ^{-1}(x_t - (A_t x_{t-1} + B_t u_t)) \nonumber  \\
 && \hspace{3cm}+\frac{\partial }{\partial x_{i,t-1}}  \frac{1}{2} (x_{t-1}-\mu _{t-1}) ^T \Sigma ^{-1}_{t-1}  (x_{t-1}-\mu _{t-1}) \\
 &=&\left( -\frac{1}{2} \frac{\partial A_{\alpha \beta } x_{\beta , t-1}}{\partial x_{i, t-1}} \right) R_{\alpha \gamma ,t } ^{-1}(x_{\gamma ,t} - (A_{\gamma \delta ,t }  x_{\delta ,t-1} + B_{\gamma \delta ,t} u_{\delta ,t})) \nonumber  \\
&&+(x_{\alpha ,t}-(A_{\alpha \beta ,t }  x_{\beta ,t-1} + B_{\alpha \beta ,t} u_{\beta ,t})R_{\alpha \gamma ,t } ^{-1} \left( -\frac{1}{2} \frac{\partial A_{\gamma \delta } x_{\delta , t-1}}{\partial x_{i, t-1}} \right)\nonumber  \\
&&-\frac{1}{2}\frac{\partial x_{\alpha ,t-1}}{\partial x_{i,t-1}}\Sigma _{\alpha \gamma,t-1} ^{-1}  (x_{\gamma ,t-1}-\mu _{\gamma ,t-1})\nonumber  \\
&&-\frac{1}{2}(x_{\alpha ,t-1}-\mu _{\alpha ,t-1})\Sigma _{\alpha \gamma,t-1} ^{-1}   \frac{\partial x_{\gamma ,t-1}}{\partial x_{i,t-1}} \hspace{1cm} \mbox{(積の微分)} \\
&=&-\frac{1}{2}A_{\alpha i}R_{\alpha \gamma ,t } ^{-1}(x_{\gamma ,t} - (A_{\gamma \delta ,t }  x_{\delta ,t-1} + B_{\gamma \delta ,t} u_{\delta ,t})) \nonumber  \\
&&-\frac{1}{2}(x_{\alpha ,t}-(A_{\alpha \beta ,t }  x_{\beta ,t-1} + B_{\alpha \beta ,t} u_{\beta ,t}))R_{\alpha \gamma ,t } ^{-1} A _{\gamma i}\nonumber  \\
&&-\frac{1}{2}\Sigma ^{-1} _{i,\gamma,t-1} (x_{\gamma ,t-1}-\mu _{\gamma ,t-1})-\frac{1}{2}(x_{\alpha ,t-1}-\mu _{\alpha ,t-1})\Sigma ^{-1}_{\alpha i,t-1} \\
&&\hspace{0.5cm} \mbox{(微分すると$\beta=i$のような項しか残らない。クロネッカーのデルタを使うと$\frac{\partial x_j}{\partial x_i} = \delta _{ij}$)} \nonumber \\
&=& -\frac{1}{2}A^T _{ i \alpha }R_{\alpha \gamma ,t } ^{-1}(x_{\gamma ,t} - (A_{\gamma \delta ,t }  x_{\delta ,t-1} + B_{\gamma \delta ,t} u_{\delta ,t})) \nonumber  \\
&&-\frac{1}{2}(x_{\alpha ,t}-(A_{\alpha \beta ,t }  x_{\beta ,t-1} + B_{\alpha \beta ,t} u_{\beta ,t}))R_{\alpha \gamma ,t } ^{-1} A _{\gamma i}\nonumber  \\
&&-\frac{1}{2}\Sigma ^{-1} _{i,\gamma,t-1} (x_{\gamma ,t-1}-\mu _{\gamma ,t-1})-\frac{1}{2}(x_{\alpha ,t-1}-\mu _{\alpha ,t-1})\Sigma ^{-1}_{\alpha i,t-1}\hspace{1cm} \mbox{($A^T _{ij} = A_{ji}$より)} \\
&=& -\frac{1}{2}A^T _{ i \alpha }R_{\alpha \gamma ,t } ^{-1}(x_{\gamma ,t} - (A_{\gamma \delta ,t }  x_{\delta ,t-1} + B_{\gamma \delta ,t} u_{\delta ,t})) \nonumber  \\
&&-\frac{1}{2} A _{\gamma i}R_{\alpha \gamma ,t } ^{-1}(x_{\alpha ,t}-(A_{\alpha \beta ,t }  x_{\beta ,t-1} + B_{\alpha \beta ,t} u_{\beta ,t}))\nonumber  \\
&&-\frac{1}{2}\Sigma ^{-1} _{i\gamma,t-1} (x_{\gamma ,t-1}-\mu _{\gamma ,t-1})-\frac{1}{2}(x_{\alpha ,t-1}-\mu _{\alpha ,t-1})\Sigma ^{-1}_{\alpha i,t-1}\nonumber\\
&&\hspace{0.5cm} \mbox{(各成分はスカラーなので可換)} \\
&=& -\frac{1}{2}A^T _{ i \alpha }R_{\alpha \gamma ,t } ^{-1}(x_{\gamma ,t} - (A_{\gamma \delta ,t }  x_{\delta ,t-1} + B_{\gamma \delta ,t} u_{\delta ,t})) \nonumber  \\
&&-\frac{1}{2} A ^T _{i \gamma }R_{\gamma \alpha ,t } ^{-1}(x_{\alpha ,t}-(A_{\alpha \beta ,t }  x_{\beta ,t-1} + B_{\alpha \beta ,t} u_{\beta ,t}))\nonumber  \\
&&-\frac{1}{2}\Sigma ^{-1} _{i\gamma ,t-1} (x_{\gamma ,t-1}-\mu _{\gamma ,t-1})-\frac{1}{2}\Sigma ^{-1}_{i \alpha ,t-1}(x_{\alpha ,t-1}-\mu _{\alpha ,t-1}) \nonumber \\
&&\hspace{0.5cm} \mbox{(共分散行列は対称なので添字を入れ替え、さらに$A$を転置。第四項も同様)} 
\end{eqnarray}
第一項と第二項は和に使われている添字が違うだけで、全く同じ項である。第三項、第四項についても同じである。結局

\begin{eqnarray}
\frac{\partial  L_t}{\partial x_{i,t-1}}  &=& -A^T _{ i \alpha }R_{\alpha \gamma ,t } ^{-1}(x_{\gamma ,t} - (A_{\gamma \delta ,t }  x_{\delta ,t-1} + B_{\gamma \delta ,t} u_{\delta ,t})) +\Sigma ^{-1}_{i \alpha}(x_{\alpha ,t-1}-\mu _{\alpha ,t-1})
\end{eqnarray}

が得られるので、行列とベクトルの積を計算してまとめてかくと
\begin{eqnarray}
\frac{\partial  L_t}{\partial x_{t-1}}  &=& -A^T _t R_t ^{-1}(x_{ t} - (A_{ t }  x_{t-1} + B_{t} u_{t})) +\Sigma ^{-1}_{t-1}(x_{t-1}-\mu _{t-1})
\end{eqnarray}

$L_t (x_{t-1},x_t)$はベクトル$x_{t-1}$のあらゆる方向に対して微分係数が0になっている点$x_{t-1} = \xi$を求めたかったのだから、上の式がゼロベクトルと等しいとした方程式を$x_{t-1}$に関して解けばよい。

実際に変形をはじめるまえにもうひとつ別の量を計算しておく。$L_t (x_{t-1},x_t)=(x_{t-1}-\xi)^T \Psi^{-1} (x_{t-1}-\xi)$の形に変形する際のあらたな共分散行列$\Psi$だ。$L_t (x_{t-1},x_t)=\frac{1}{2}(x_{t-1}-\xi)^T \Psi^{-1} (x_{t-1}-\xi)$に変形できることを仮定すると、この行列は以下のように計算できる

\begin{eqnarray}
\Psi _{ij} ^{-1 }&=& \frac{\partial ^2 L_t (x_{t-1},x_t)}{\partial x_{i,t-1}\partial x_{j,t-1}} =\frac{\partial ^2 L_t}{\partial x_{i,t-1}\partial x_{j,t-1}}\\ &&\hspace{0.5cm} \mbox{($L_t (x_t)$は$x_{t-1}$で微分してもゼロ)}\\
&=&\frac{1}{2} \frac{\partial ^2 }{\partial x_{i,t-1}\partial x_{j,t-1}} (x_{t-1}-\xi)^T \Psi^{-1} (x_{t-1}-\xi) \\
&=& \frac{1}{2} \frac{\partial }{\partial x_{i,t-1}}  \Psi^{-1}_{j \alpha } (x_{\alpha t-1}-\xi _\alpha) +\frac{1}{2} \frac{\partial }{\partial x_{i,t-1}}  (x_{\alpha t-1}-\xi _\alpha)^T \Psi^{-1}_{\alpha j}  \\
&=& \frac{1}{2}(\Psi ^{-1}_{ij}+\Psi ^{-1}_{ji})
\end{eqnarray}
$\Psi$は定義から共分散行列で対称行列で、対称行列の逆行列はまた対称行列になる($1=(AA^{-1})^T=(A^{-1})^TA$だから、$A^{-1}=(A^{-1})^T$)ので両辺は等しい。なので$\Psi$を求めるため、$L_t$をさらにもう一回微分する。

\begin{eqnarray}
\frac{\partial ^2  L_t}{\partial x_{t-1}^2}  &=& \frac{\partial }{\partial x_{t-1}}-A^T_t R_t ^{-1}(x_{ t} - (A_{ t }  x_{t-1} + B_{t} u_{t})) +\Sigma ^{-1}_{t-1}(x_{t-1}-\mu _{t-1}) \\
&=&\frac{\partial }{\partial x_{t-1}}A^T R_t ^{-1} A_t x_{t-1}+\frac{\partial }{\partial x_{t-1}}\Sigma ^{-1}_{t-1}x_{t-1} \\
&=& A^T R_t ^{-1} A_t+\Sigma ^{-1}_{t-1}=\Psi _t ^{-1}
\end{eqnarray}

それではいよいよ$L_t (x_{t-1},x_t)$の$x_{t-1}$についてあらゆる方向に対して微分係数が0になっている点$x_{t-1} = \xi$を求める。

\begin{eqnarray}
\frac{\partial  L_t}{\partial x_{t-1}}  &=& -A^T _t R_t ^{-1}(x_{ t} - (A_{ t }  x_{t-1} + B_{t} u_{t})) +\Sigma ^{-1}_{t-1}(x_{t-1}-\mu _{t-1})=0 \\
&&-A^T_t R_t^{-1} x_t + A^T_t R_t^{-1} A_t x_{t-1} + A^T_t R_t^{-1} B_t u_t + \Sigma ^{-1}_{t-1} x_{t-1} - \Sigma ^{-1}_{t-1} \mu _{t-1}=0\\
&&(A^T_t R_t^{-1} A_t +  \Sigma ^{-1}_{t-1})x_{t-1 } = A^T_t R_t^{-1} x_t -A^T_t R_t^{-1} B_t u_t +\Sigma ^{-1}_{t-1} \mu _{t-1} \nonumber \\
&&\hspace{0.5cm}\mbox{($x_{t-1}$に関する項を左辺にまとめた)}\\
&&\Psi _t x_{t-1} ^{-1} x_{t-1} =A^T_t R_t^{-1}(x_t -B_t u_t )+\Sigma ^{-1}_{t-1} \mu _{t-1}  \hspace{0.5cm}\mbox{($\Psi$の定義)}\\
&&x_{t-1}=\Psi _t[A^T_t R_t^{-1}(x_t -B_t u_t )+\Sigma ^{-1}_{t-1} \mu _{t-1}  ]
\end{eqnarray}

これで無事、$L_t(x_{t-1},x_t)$としてほしい形が求まった。すなわち

\begin{eqnarray}
&&L_t(x_{t-1},x_t)\\
&=&\frac{1}{2}(x_{t-1}-\xi) ^T \Psi _t ^{-1}(x_{t-1}-\xi) \\
&=&\frac{1}{2}(x_{t-1}-\Psi _t[A^T_t R_t^{-1}(x_t -B_t u_t )+\Sigma ^{-1}_{t-1} \mu _{t-1}  ]) ^T \Psi _t ^{-1} \nonumber \\
&&\hspace{2cm}(x_{t-1}-\Psi _t[A^T_t R_t^{-1}(x_t -B_t  u_t )+\Sigma ^{-1}_{t-1} \mu _{t-1}  ]) 
\end{eqnarray}

である。実際ここで求めた$L_t(x_{t-1},x_t)$を使って$L_t (x_t) = L_t - L_t(x_{t-1},x_t)$が$x_{t-1}$を含まない形になっているか確認しておく。本書の通りこれには具体的に計算してみてもよいが、$\frac{\partial}{\partial x_{t-1}} [L_t - L_t(x_{t-1},x_t)]$が$x_{t-1}$に関して恒等的に0に等しいことを示す方法もある。

このノートでは$\frac{\partial L_t}{\partial x_{t-1}}$と$\frac{\partial L_t(x_{t-1},x_t)}{\partial x_{t-1}} $が等しいことを示すことで、上記の事実を示す。$\frac{\partial L_t}{\partial x_{t-1}}$は前に計算したので、$\frac{\partial L_t}{\partial x_{t-1}}$を計算すればよい。

\begin{eqnarray}
\frac{\partial L_t}{\partial x_{t-1}} &=&\frac{\partial}{\partial x_{t-1}} \frac{1}{2}(x_{t-1}-\xi) ^T \Psi _t ^{-1}(x_{t-1}-\xi) \\
&=&\frac{1}{2}\frac{\partial x_{t-1}}{\partial x_{t-1}}^T \Psi _t ^{-1}(x_{t-1}-\xi) +\frac{1}{2} (x_{t-1}-\xi) ^T \Psi _t ^{-1}\frac{\partial x_{t-1}}{\partial x_{t-1}}\hspace{0.2cm } \mbox{(積の微分、ch0のノート参照)}\\
&=& \Psi _t ^{-1}(x_{t-1}-\xi) \hspace{0.5cm } \mbox{($\Psi は対称行列$)} \\
&=& \Psi _t ^{-1}x_{t-1}-\Psi _t ^{-1} \Psi _t[A^T_t R_t^{-1}(x_t -B_t u_t )+\Sigma ^{-1}_{t-1} \mu _{t-1}  ] \hspace{0.2cm } \mbox{($\xi$の定義)}\\
&=& (A^T R_t ^{-1} A_t+\Sigma ^{-1}_{t-1})x_{t-1}-A^T_t R_t^{-1}(x_t -B_t u_t )-\Sigma ^{-1}_{t-1} \mu _{t-1}  \hspace{0.2cm } \mbox{($\Psi$の定義)}\\
&=& -A^T R_t ^{-1}(x_t -(A_t x_{t-1} +B_t u_t)) + \Sigma ^{-1}_{t-1}(x_{t-1}-u_{t-1})
\end{eqnarray}
これは$\frac{\partial L_t}{\partial x_{t-1}}$と等しいので、$\frac{\partial}{\partial x_{t-1}} [L_t - L_t(x_{t-1},x_t)]$は0(定数関数)となる。よって$L_t - L_t(x_{t-1},x_t)$は$x_{t-1}$によらない。

これで無事に$L_t$を分解できた。信念の計算に戻ると$\int \exp L_t (x_{t-1},x_t)dx_{t-1}$を定数にとることができたので

\begin{eqnarray}
\overline{bel}(x_t) &=& \eta \exp[-L_t (x_t)]  \int \exp [-L_t (x_{t-1},x_t)]dx_{t-1}\\
&=& \eta \exp [- L_t (x_t)]
\end{eqnarray}

と簡単に書き直せる。1行目と2行目で$\eta$を取り替えている。信念の計算には$L_t (x_t)$が必要である。せっかくなので$L_t$を愚直に展開して$L_t (x_{t-1},x_t)$を引く以外の方法で計算してみる。

$L_t (x_t)$は$x_{t-1} $によらないので$L_t - L_t (x_{t-1},x_t)$はどの$x_{t-1}$で計算してもよい。なので$x_{t-1}=\xi$で計算すると、$L_t(x_{t-1},x_t)=0$なので、展開して差し引くまでもまく$L_t(x_t) = L_t$になる。つまり、$L_t$に$x_{t-1}=\xi $を代入すればよい。

\begin{eqnarray}
L_t(x_t) &=& \frac{1}{2}(x_t - (A_t \xi + B_t u_t))^T R_t ^{-1}(x_t - (A_t \xi + B_t u_t)) \nonumber  \\
 && \hspace{3cm}+\frac{1}{2} (\xi-\mu _{t-1})^T  \Sigma ^{-1}_t  (\xi-\mu _{t-1})
\end{eqnarray}




\end{document}